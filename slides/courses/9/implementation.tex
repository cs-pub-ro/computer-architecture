

% Slide: Dependency on Main Memory
\begin{frame}
    \frametitle{Dependency on Main Memory}
    \begin{itemize}
        \item \textbf{Independent Control Memory}: Separate memory specifically for microinstructions, allowing fast access.
        \item \textbf{Shared Control Memory}: Microinstructions share space with main memory, creating possible contention.
        \item \textbf{Cached Control Memory}: Microinstructions are cached from main memory, improving access speed with fewer memory calls.
    \end{itemize}
    \note{
        This slide discusses the organizational impact of where control memory is stored.
        Each structure offers trade-offs between speed and design complexity.
        For example, independent memory provides high-speed access,
        whereas shared memory may lead to access delays.
    }
\end{frame}

% Slide: Control Memory Organization
\begin{frame}
    \frametitle{Control Memory Organization}
    \begin{itemize}
        \item \textbf{Word-Addressed}: Each address points to a single microinstruction.
        \item \textbf{Page-Addressed}: Each address points to a set of microinstructions in different pages (e.g., branching or subroutine instructions).
        \item \textbf{Block-Addressed}: Each address points to a block of related microinstructions, reducing fetches for grouped operations.
        \item \textbf{Divided Memory}: Unique microinstructions stored separately; control memory holds references.
        \item \textbf{Two-Level Memory}: Microinstructions point to nano-instructions, which further specify control signals.
    \end{itemize}
    \note{
        Different control memory organizations allow trade-offs in memory usage and control complexity.
        Page and block organization allow faster execution for sequences of instructions,
        while divided and two-level organizations reduce redundancy and can handle complex microoperations.
    }
\end{frame}

\begin{frame}
    \frametitle{Micro-instruction Format}
    \textbf{Constraints:}
    \begin{itemize}
        \item Parallelism between micro-operations.
        \item Control signals for the data flow.
        \item Felixibility for future expansion.
        \item Micro-instruction size.
    \end{itemize}
\end{frame}

% Slide: Micro-instruction Format
\begin{frame}
    \frametitle{Micro-instruction Format}
    \textbf{Key Considerations:}
    \begin{itemize}
        \item \textbf{Parallelism}: Allow simultaneous execution of multiple micro-operations.
        \item \textbf{Control Signal Flexibility}: Supports data flow, branch control, and additional operations for the underlying hardware.
        \item \textbf{Future Expansion}: Design accommodates new instructions or signals.
        \item \textbf{Micro-instruction Size}: Optimizes between memory usage and instruction capability.
    \end{itemize}
    \note{
        Micro-instruction format impacts CPU performance and expandability.
        Parallelism and control signal flexibility are critical for efficient data flow,
        while expandability ensures the design remains adaptable for future updates.
    }
\end{frame}

% Slide: Types of Micro-instruction Coding
\begin{frame}
    \frametitle{Types of Micro-instruction Coding}
    \begin{itemize}
        \item \textbf{Horizontal Coding}: Each bit represents a micro-operation, enabling high parallelism.
        \item \textbf{Vertical Coding}: Each micro-instruction represents a single micro-operation.
        \item \textbf{Minimal Coding}: Fields group compatible micro-operations (horizontal and vertical coding combined).
        \item \textbf{Residual Coding}: Micro-operations are in control registers, modified by microinstructions.
        \item \textbf{Address Coding}: Each micro-instruction is an refference for a unique micro-operation in a separate memory.
        \item \textbf{Mixed Coding}: Combines field-based micro-operations with address-based sequential steps.
    \end{itemize}
    \note{
        Each coding type offers different levels of flexibility and memory efficiency.
        For example, horizontal coding is suited for CPUs needing high parallelism,
        while address and residual coding enable efficient memory use by referencing existing control registers or memory locations.
        Mixed coding - First part of the micro-instruction
        is form of fields which contains coded micro-operations. The second
        part is an address to the next microinstruction.
    }
\end{frame}

% Slide: Execution of Micro-instructions
\begin{frame}
    \frametitle{Execution of Micro-instructions}
    Each micro-instruction is executed in two stages: fetch and execute.
    \begin{itemize}
        \item \textbf{Sequential Execution}: Microinstructions execute in a fixed order, ideal for simple instruction flow.
        \item \textbf{Parallel Execution}: The next microinstruction is fetched while the current one executes, improving throughput.
        \item \textbf{Sequential-Parallel Execution}: Instructions run in parallel except for branches, balancing speed with control complexity.
    \end{itemize}
    \note{
        This slide introduces students to different execution models for micro-instructions.
        Sequential-parallel execution is a common optimization,
        allowing multiple instructions to run simultaneously while handling branches independently.
    }
\end{frame}

% Slide: Phases of Micro-instruction Execution
\begin{frame}
    \frametitle{Phases of Micro-instruction Execution}
    \begin{itemize}
        \item \textbf{Mono-phase Execution}: All micro-operations run in a single phase.
        \item \textbf{Poly-phase Execution}: Micro-operations execute across different phases, allowing complex singals sequencing.
    \end{itemize}
    \textbf{Cycle Time}: Execution can have fixed or variable cycle times depending on the micro-instruction complexity.
    \note{
        Mono-phase and poly-phase executions highlight flexibility in instruction handling.
        Mono-phase is simpler but less versatile than poly-phase,
        which can handle more intricate operations at the cost of increased cycle time variability.
    }
\end{frame}

