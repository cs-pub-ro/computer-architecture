%\documentclass[notes,usenames,dvipsnames]{beamer}
%\documentclass[notes]{beamer}       % print frame + notes
%\documentclass[notes=only]{beamer}   % only notes
\documentclass[usenames,dvipsnames]{beamer}             % only frames
\usepackage{pgfpages}
%\setbeameroption{show notes on second screen}
%\setbeameroption{show notes}
%subfigures
\usepackage{caption}
\usepackage{subcaption}

%tables packages
\usepackage{multirow}

% math
\usepackage{amsmath}

% bash command
\usepackage{graphicx}
\usepackage{listings}
% varbatim for ascii figures
\usepackage[T1]{fontenc}
\usepackage[utf8]{inputenc}
%\usepackage{verbatim}
\usepackage{lipsum} % for context
\usepackage{fancyvrb}
\usepackage{varwidth}


\newsavebox{\asciigcn}

% notes prefixed in pympress
\addtobeamertemplate{note page}{}{\thispdfpagelabel{notes:\insertframenumber}}
%theme used
\usetheme{Madrid}
%Information to be included in the title page:
\title[Computer Architecture] %optional
{Arhitectura Calculatoarelor}

\subtitle{Seria AB - Curs 1 - Structură Calculator Numeric}

\author[Ștefan-Dan Ciocîrlan] % (optional, for multiple authors)
{}

\institute[NUSTPB] % (optional)
{
  \inst{}%
  National University of Science and Technology\\
  POLITEHNICA Bucharest
}

\date[NUSTPB 2024] % (optional)
{Computer Architecture}

\logo{
\includegraphics[height=1.3cm]{../../media/logo-poli-color9.png}
%\includegraphics[height=0.7cm]{media/sbipLogo.pdf}
}


% Roman numerals
\newcommand*{\rom}[1]{\expandafter\@slowromancap\romannumeral #1@}
%split
\usepackage{amsmath}
%colors
%\usepackage[usenames,dvipsnames]{color} %loaded by the dcoument class
%subfigure
\usepackage{subcaption}
% block over block uncover
\setbeamercovered{invisible}
%\setbeamercovered{transparent}

%extra slide content
\AtBeginSection[]
{
  \begin{frame}
    \frametitle{Conținut}
    \tableofcontents[currentsection]
  \end{frame}
}
%notes or not
%\setbeamertemplate{note page}[plain]

\begin{document}

\frame{\titlepage}


\section{Calculator - Sistem de prelucrare a informației}
\begin{frame}
    \frametitle{Information Representation}
    \begin{itemize}
        \item Decimal
        \item Binary
        \item Hexadecimal
    \end{itemize}
    \note{
    }
\end{frame}

\begin{frame}
    \frametitle{Information Representation}

    \begin{table}[h!]
        \centering
        \scalebox{0.8}{
        \begin{tabular}{|c|c|c|}
            \hline
            \textbf{Decimal} & \textbf{Binary} & \textbf{Hexadecimal} \\
            \hline
            0 & 0000 & 0x0 \\
            1 & 0001 & 0x1 \\
            2 & 0010 & 0x2 \\
            3 & 0011 & 0x3 \\
            4 & 0100 & 0x4 \\
            5 & 0101 & 0x5 \\
            6 & 0110 & 0x6 \\
            7 & 0111 & 0x7 \\
            8 & 1000 & 0x8 \\
            9 & 1001 & 0x9 \\
            10 & 1010 & 0xA \\
            11 & 1011 & 0xB \\
            12 & 1100 & 0xC \\
            13 & 1101 & 0xD \\
            14 & 1110 & 0xE \\
            15 & 1111 & 0xF \\
            \hline
        \end{tabular}
        }
        \caption{Decimal, Binary, and Hexadecimal Values}
        \label{tab:binary_decimal_hexadecimal}
    \end{table}

    \note{
    }
\end{frame}


\begin{frame}
    \frametitle{Information Representation}
\end{frame}

\begin{frame}
    \frametitle{Big Endian vs Little Endian}
    \begin{itemize}
        \item Big Endian
        \begin{itemize}
            \item Most significant byte first
            \item Network byte order
            \item Example: 0x12345678 is stored as 0x12 0x34 0x56 0x78
        \end{itemize}
        \item Little Endian
        \begin{itemize}
            \item Least significant byte first
            \item Intel byte order
            \item Example: 0x12345678 is stored as 0x78 0x56 0x34 0x12
        \end{itemize}
    \end{itemize}
\end{frame}




\begin{frame}
    \frametitle{Exam Question}
    Given the following hexadecimal number $0x3A2E$ in big endian format, convert it to binary and decimal.
    $0x3A2E = 0011\ 1010\ \ 0010\ 1110 = (3 * 16 + 10) * 2^8 + (2 * 16 + 14) = 58 * 256 + 46 = 14894$
\end{frame}


\section{Model teoretic al Calculatorului Numeric}
\begin{frame}
    \frametitle{Expersii Regulate/Logică combinațională}
    \newsavebox{\asciiregular}
    \begin{lrbox}{\asciiregular}
        \begin{varwidth}{\maxdimen}
        \VerbatimInput[fontsize=\scriptsize]{media/regular.ascii}
        \end{varwidth}
    \end{lrbox}%

    \begin{figure}[h]
        \centering
        \scalebox{0.8}{\usebox{\asciiregular}}
    \end{figure}
    \begin{itemize}
        \item Expresii regulate $O = [a-zA-Z0-9]+$
        \item Logică combinațională $O = i_0 \oplus i_1$
        \item Are nevoie de toată intrarea pentru a produce ieșirea
    \end{itemize}
\end{frame}

\begin{frame}
    \frametitle{Automat Finite}
    \newsavebox{\asciifsm}
    \begin{lrbox}{\asciifsm}
        \begin{varwidth}{\maxdimen}
        \VerbatimInput[fontsize=\scriptsize]{media/fsm.ascii}
        \end{varwidth}
    \end{lrbox}%

    \begin{figure}[h]
        \centering
        \scalebox{0.8}{\usebox{\asciifsm}}
    \end{figure}
    \begin{itemize}
        \item Are o stare internă ce poate fi modificată
        \item Nu poate accessa intrări anterior anterior
    \end{itemize}
\end{frame}


\begin{frame}
    \frametitle{Automat cu stivă/pushdown}
    \newsavebox{\asciistackfsm}
    \begin{lrbox}{\asciistackfsm}
        \begin{varwidth}{\maxdimen}
        \VerbatimInput[fontsize=\scriptsize]{media/stackfsm.ascii}
        \end{varwidth}
    \end{lrbox}%

    \begin{figure}[h]
        \centering
        \scalebox{0.8}{\usebox{\asciistackfsm}}
    \end{figure}
    \begin{itemize}
        \item Are o stivă unde poate stoca și re accesa intrările.
        \item Stiva este limitată, o intrare citită salvată și folosită nu poate fi folosită din nou.
    \end{itemize}
\end{frame}


\begin{frame}
    \frametitle{Model Turing Machine}
    \newsavebox{\asciituring}
    \begin{lrbox}{\asciituring}
        \begin{varwidth}{\maxdimen}
        \VerbatimInput[fontsize=\scriptsize]{media/turing.ascii}
        \end{varwidth}
    \end{lrbox}%

    \begin{figure}[h]
        \centering
        \scalebox{0.8}{\usebox{\asciituring}}
    \end{figure}
    \begin{itemize}
        \item Are o rolă infinită și poate reaccesa intrările și ieșirle anterioare.
    \end{itemize}
\end{frame}

\section{Model structural al Calculatorului Numeric}
\begin{frame}
    \frametitle{Proprietați CN}
\begin{itemize}
    \item Determinist în teorie, predictibil în inginerie cu eroare neglijabilă ce poate fi modelată.
    \item Modelat după Turing Machine.
\end{itemize}

\end{frame}

\begin{frame}
    \frametitle{Model general de CN}
    \begin{lrbox}{\asciigcn}
        \begin{varwidth}{\maxdimen}
        \VerbatimInput[fontsize=\scriptsize]{media/gcn.ascii}
        \end{varwidth}
    \end{lrbox}%

    \begin{figure}[h]
        \centering
        \scalebox{0.9}{\usebox{\asciigcn}}
    \end{figure}

    \begin{itemize}
        \item Diferențe între reprezentarea internă și externă a datelor.
    \end{itemize}
\end{frame}


\begin{frame}
    \frametitle{Model structural al CN}
    \newsavebox{\asciimscn}
    \begin{lrbox}{\asciimscn}
        \begin{varwidth}{\maxdimen}
        \VerbatimInput[fontsize=\scriptsize]{media/mscn.ascii}
        \end{varwidth}
    \end{lrbox}%

    \begin{figure}[h]
        \centering
        \scalebox{0.8}{\usebox{\asciimscn}}
    \end{figure}
    \begin{itemize}
        \item Diferneță între echipamentele periferice și calculatorul numeric.
        \item Memorie internă și unitatea centrală de procesare.
    \end{itemize}
\end{frame}


\begin{frame}
    \frametitle{Structura unui CN}
    \newsavebox{\asciistructcn}
    \begin{lrbox}{\asciistructcn}
        \begin{varwidth}{\maxdimen}
        \VerbatimInput[fontsize=\scriptsize]{media/structcn.ascii}
        \end{varwidth}
    \end{lrbox}%

    \begin{figure}[h]
        \centering
        \scalebox{0.65}{\usebox{\asciistructcn}}
    \end{figure}
    \begin{itemize}
        \item Ierarhie de memorie.
        \item Control și procesare.
    \end{itemize}
\end{frame}


\begin{frame}
    \frametitle{Structura unui Sistem de Calcul (SC)}
    \newsavebox{\asciistructsoc}
    \begin{lrbox}{\asciistructsoc}
        \begin{varwidth}{\maxdimen}
        \VerbatimInput[fontsize=\scriptsize]{media/structsoc.ascii}
        \end{varwidth}
    \end{lrbox}%

    \begin{figure}[h]
        \centering
        \scalebox{0.47}{\usebox{\asciistructsoc}}
    \end{figure}
    \begin{itemize}
        \item Control memorie.
    \end{itemize}
\end{frame}

\section{Model funcțional al Calculatorului Numeric}
\begin{frame}
    \frametitle{Niveluri funcționale ale unui CN}
\begin{enumerate}
    \item Dispozitve și circuite electronice (Tranzistoare, porți logice) - hardware
    \item Unități funcționale (UAL, Memorie, Interfețe) - firmware (Micropgramare) + hardware
    \item Mașină fizică Strcuturi de interconectare și echipamente periferice - hardware
    \item Nucleul sistem de operare - BIOS - firmware
    \item Sistem de Operare și API-ul său - software
    \item Limbaj de programare și compilator - software
    \item Aplicații și biblioteci de programare - software
    \item Interfața cu utilizatorul - software
\end{enumerate}

\end{frame}

\begin{frame}
    \frametitle{Hardware vs Software}
\begin{itemize}
    \item Anumite funcții pot fi implementate atât în hardware cât și în software
    \item Decizia de a implementa o funcție în hardware sau software se bazează pe:
    \begin{itemize}
        \item Costul implementării
        \item Viteza de execuție
        \item Raport cost/performanță
        \item Siguranță în funcționare
        \item Frecvența unor modificări
    \end{itemize}
    \item Tendința actuală este de a trece cât mai multe funcții în hardware
\end{itemize}

\end{frame}


\section{Unități funcționale a Calculatorului Numeric}
\begin{frame}
    \frametitle{Register}
    \begin{circuitikz}
        \draw
        (0,0) node[draw, minimum width=2cm, minimum height=3cm] (chip) {Register}
        (chip.west) ++(-0.5,1) node[left] {Data Input (DI)} -- (chip.west |- 0,1)
        (chip.west) ++(-0.5,0) node[left] {Write Enable (WE)} -- (chip.west |- 0,0)
        (chip.west) ++(-0.5,-1) node[left] {Output Enable (OE)} -- (chip.west |- 0,-1)
        (chip.east) ++(0.5,0) node[right] {Data Output (DO)} -- (chip.east |- 0,0);
    \end{circuitikz}
    
    \note{
    }
\end{frame}

\begin{frame}
    \frametitle{General Purpose Registers (GR)}
    \begin{table}[]
        \begin{tabular}{|l|l|l|}
            \hline
            \textbf{Register} & \textbf{Acronym} & \textbf{Size} \\ \hline
            Register A & RA & 16-bit \\ \hline
            Register B & RB & 16-bit \\ \hline
            Register C & RC & 16-bit \\ \hline
            Stack Pointer Register & SP & 16-bit \\ \hline
            Index Register A & XA & 16-bit \\ \hline
            Index Register B & XB & 16-bit \\ \hline
            Base Address A & BA & 16-bit \\ \hline
            Base Address B & BB & 16-bit \\ \hline
        \end{tabular}
    \end{table}
    \note{
    }
\end{frame}


\begin{frame}
    \frametitle{Registers File (RF)}
    
    \begin{circuitikz}
        % Demux
        \draw
        (0,0) node[muxdemux, muxdemux def={Lh=4, Rh=8, NL=3, NR=8, NT=0, NB=3, w=2,
        square pins=1}] (C) at (0,0) {I};
        % Inputs
    \node[left, font=\tiny] at (C.lpin 1) {DI};
    \node[left, font=\tiny] at (C.lpin 2) {WE};
    \node[left, font=\tiny] at (C.lpin 3) {OE};
    
    % Outputs
    \node[above, font=\tiny] at (C.rpin 1) {RA};
    \node[right, font=\tiny] at (C.rpin 2) {RB};
    \node[right, font=\tiny] at (C.rpin 3) {RC};
    \node[right, font=\tiny] at (C.rpin 4) {SP};
    \node[right, font=\tiny] at (C.rpin 5) {XA};
    \node[right, font=\tiny] at (C.rpin 6) {XB};
    \node[right, font=\tiny] at (C.rpin 7) {BA};
    \node[right, font=\tiny] at (C.rpin 8) {BB};
    \node[below, font=\tiny] at (C.bpin 1) {Register Address (REG)};
    
    % Split RA output into 3 connections
    \draw (C.rpin 1) -- ++(1.0,0) coordinate (split) 
    (split) |- ++(1,1.0) node[right, font=\tiny] {DI}
    (split) |- ++(1,-1.0) node[right, font=\tiny] {WE}
    (split) -- ++(1,0) node[right, font=\tiny] {OE};
    
    % Connect to another component
    \draw (split) -- ++(1,0) node[draw, minimum width=2cm, minimum height=3cm, anchor=west] (chip) {RA}
    (chip.east) node[left, font=\tiny] {DO}
    (chip.east) |- ++(1,0.0) node[muxdemux, muxdemux def={Lh=8, Rh=1, NL=8, NR=1, NT=0, NB=3, w=2,
    square pins=1}] (D) at (6.5,0) {O};
    % Inputs
    \node[above, font=\tiny] at (D.lpin 1) {RA};
    \node[left, font=\tiny] at (D.lpin 2) {RB};
    \node[left, font=\tiny] at (D.lpin 3) {RC};
    \node[left, font=\tiny] at (D.lpin 4) {SP};
    \node[left, font=\tiny] at (D.lpin 5) {XA};
    \node[left, font=\tiny] at (D.lpin 6) {XB};
    \node[left, font=\tiny] at (D.lpin 7) {BA};
    \node[left, font=\tiny] at (D.lpin 8) {BB};
    \node[below, font=\tiny] at (D.bpin 1) {Register Address (REG)};

    % Outputs
    \node[above, font=\tiny] at (D.rpin 1) {DO};

    \end{circuitikz}
\end{frame}

\begin{frame}
    \frametitle{Special Purpose Registers}
    \begin{table}[]
        \begin{tabular}{|l|l|l|}
            \hline
            \textbf{Register} & \textbf{Acronym} & \textbf{Size} \\ \hline
            Program Counter & PC & 16-bit \\ \hline
            Instruction Register & IR & 16-bit \\ \hline
            Memory Address Register & MA & 16-bit \\ \hline
            Flags Register & FR & 16-bit \\ \hline
            Operand Register 1 & T1 & 16-bit \\ \hline
            Operand Register 2 & T2 & 16-bit \\ \hline
            Input/Output Addressing Register & IOA & 16-bit \\ \hline
        \end{tabular}
    \end{table}
\end{frame}

\begin{frame}
    \frametitle{Memory (M)}
    \begin{itemize}
        \item Address width: 16-bit
        \item Data width: 16-bit
    \end{itemize}
    \begin{circuitikz}
        \draw
        (0,0) node[draw, minimum width=2cm, minimum height=4cm] (chip) {Memory}
        (chip.west) ++(-0.5,1) node[left] {Address (MA)} -- (chip.west |- 0,1)
        (chip.west) ++(-0.5,0.5) node[left] {Memory Input (MI)} -- (chip.west |- 0,0.5)
        (chip.west) ++(-0.5,-0.5) node[left] {Write Enable (WE)} -- (chip.west |- 0,-0.5)
        (chip.west) ++(-0.5,-1) node[left] {Output Enable (OE)} -- (chip.west |- 0,-1)
        (chip.east) ++(0.5,0) node[right] {Memory Output (MO)} -- (chip.east |- 0,0);
    \end{circuitikz}

\end{frame}

\begin{frame}
    \frametitle{Arithmetic Logic Unit (ALU)}
    \begin{circuitikz}
        \draw
        (0,0) node[muxdemux, muxdemux def={Lh=7, NL=2, Rh=4, NR=2, NB=1, NT=2, w=4,
        inset w=1, inset Lh=2, inset Rh=0, square pins=1}] (chip) {ALU}
        % Inputs
        (chip.lpin 1) node[left] {Operand 1 (T1)}
        (chip.lpin 2) node[left] {Operand 2 (T2)}
        (chip.bpin 1) node[below] {Operation (OP)}
        (chip.tpin 1) node[left] {Carry In (CI)}
        (chip.tpin 2) node[right] {Output Enable (OE)}

        % Outputs
        (chip.rpin 1) node[right] {Result (R)}
        (chip.rpin 2) node[right] {Flags (FR)};

    \end{circuitikz}
\end{frame}

\begin{frame}
    \frametitle{Internal Bus}
    \begin{table}[]
        \resizebox{\textwidth}{!}{%
        \begin{tabular}{|l|l|l|l|l|l|l|l|l|l|l|l|}
            \hline
            \textbf{Source} & \textbf{GR} & \textbf{M} & \textbf{T1} & \textbf{T2} & \textbf{IR} & \textbf{PC} & \textbf{IO} & \textbf{IOA} & \textbf{ALU} & \textbf{MA} & \textbf{FR} \\ \hline
            \textbf{GR} & X & X & X & X & - & X & X & - & - & X & - \\ \hline
            \textbf{M} & X & X & X & X & X & X & X & - & - & X & X \\ \hline
            \textbf{T1} & - & - & - & - & - & - & - & - & X & - & - \\ \hline
            \textbf{T2} & - & - & - & - & - & - & - & - & X & - & - \\ \hline
            \textbf{IR} & - & - & - & - & - & - & - & X & - & - & - \\ \hline
            \textbf{PC} & X & X & - & - & - & X & - & - & - & X & - \\ \hline
            \textbf{IO} & X & X & - & - & - & - & - & - & - & - & - \\ \hline
            \textbf{IOA} & - & - & - & - & - & - & X & - & - & - & - \\ \hline
            \textbf{ALU} & X & X & X & X & - & X & - & - & - & X & X \\ \hline
            \textbf{MA} & - & - & - & - & - & - & - & - & - & - & - \\ \hline
            \textbf{FR} & - & X & - & - & - & - & - & - & - & - & - \\ \hline
        \end{tabular}
        }
    \end{table}
    
\end{frame}

\section{Q\&A}
\begin{frame}
\end{frame}

%\begin{frame}
%\frametitle{Table of Contents}
%\tableofcontents
%\end{frame}


\end{document}