\begin{frame}
    \frametitle{Register}
    \begin{circuitikz}
        \draw
        (0,0) node[draw, minimum width=2cm, minimum height=3cm] (chip) {Register}
        (chip.west) ++(-0.5,1) node[left] {Data Input (DI)} -- (chip.west |- 0,1)
        (chip.west) ++(-0.5,0) node[left] {Write Enable (WE)} -- (chip.west |- 0,0)
        (chip.west) ++(-0.5,-1) node[left] {Output Enable (OE)} -- (chip.west |- 0,-1)
        (chip.east) ++(0.5,0) node[right] {Data Output (DO)} -- (chip.east |- 0,0);
    \end{circuitikz}
    
    \note{
    }
\end{frame}

\begin{frame}
    \frametitle{General Purpose Registers (GR)}
    \begin{table}[]
        \begin{tabular}{|l|l|l|}
            \hline
            \textbf{Register} & \textbf{Acronym} & \textbf{Size} \\ \hline
            Register A & RA & 16-bit \\ \hline
            Register B & RB & 16-bit \\ \hline
            Register C & RC & 16-bit \\ \hline
            Stack Pointer Register & SP & 16-bit \\ \hline
            Index Register A & XA & 16-bit \\ \hline
            Index Register B & XB & 16-bit \\ \hline
            Base Address A & BA & 16-bit \\ \hline
            Base Address B & BB & 16-bit \\ \hline
        \end{tabular}
    \end{table}
    \note{
    }
\end{frame}


\begin{frame}
    \frametitle{Registers File (RF)}
    
    \begin{circuitikz}
        % Demux
        \draw
        (0,0) node[muxdemux, muxdemux def={Lh=4, Rh=8, NL=3, NR=8, NT=0, NB=3, w=2,
        square pins=1}] (C) at (0,0) {I};
        % Inputs
    \node[left, font=\tiny] at (C.lpin 1) {DI};
    \node[left, font=\tiny] at (C.lpin 2) {WE};
    \node[left, font=\tiny] at (C.lpin 3) {OE};
    
    % Outputs
    \node[above, font=\tiny] at (C.rpin 1) {RA};
    \node[right, font=\tiny] at (C.rpin 2) {RB};
    \node[right, font=\tiny] at (C.rpin 3) {RC};
    \node[right, font=\tiny] at (C.rpin 4) {SP};
    \node[right, font=\tiny] at (C.rpin 5) {XA};
    \node[right, font=\tiny] at (C.rpin 6) {XB};
    \node[right, font=\tiny] at (C.rpin 7) {BA};
    \node[right, font=\tiny] at (C.rpin 8) {BB};
    \node[below, font=\tiny] at (C.bpin 1) {Register Address (REG)};
    
    % Split RA output into 3 connections
    \draw (C.rpin 1) -- ++(1.0,0) coordinate (split) 
    (split) |- ++(1,1.0) node[right, font=\tiny] {DI}
    (split) |- ++(1,-1.0) node[right, font=\tiny] {WE}
    (split) -- ++(1,0) node[right, font=\tiny] {OE};
    
    % Connect to another component
    \draw (split) -- ++(1,0) node[draw, minimum width=2cm, minimum height=3cm, anchor=west] (chip) {RA}
    (chip.east) node[left, font=\tiny] {DO}
    (chip.east) |- ++(1,0.0) node[muxdemux, muxdemux def={Lh=8, Rh=1, NL=8, NR=1, NT=0, NB=3, w=2,
    square pins=1}] (D) at (6.5,0) {O};
    % Inputs
    \node[above, font=\tiny] at (D.lpin 1) {RA};
    \node[left, font=\tiny] at (D.lpin 2) {RB};
    \node[left, font=\tiny] at (D.lpin 3) {RC};
    \node[left, font=\tiny] at (D.lpin 4) {SP};
    \node[left, font=\tiny] at (D.lpin 5) {XA};
    \node[left, font=\tiny] at (D.lpin 6) {XB};
    \node[left, font=\tiny] at (D.lpin 7) {BA};
    \node[left, font=\tiny] at (D.lpin 8) {BB};
    \node[below, font=\tiny] at (D.bpin 1) {Register Address (REG)};

    % Outputs
    \node[above, font=\tiny] at (D.rpin 1) {DO};

    \end{circuitikz}
\end{frame}

\begin{frame}
    \frametitle{Special Purpose Registers}
    \begin{table}[]
        \begin{tabular}{|l|l|l|}
            \hline
            \textbf{Register} & \textbf{Acronym} & \textbf{Size} \\ \hline
            Program Counter & PC & 16-bit \\ \hline
            Instruction Register & IR & 16-bit \\ \hline
            Memory Address Register & MA & 16-bit \\ \hline
            Flags Register & FR & 16-bit \\ \hline
            Operand Register 1 & T1 & 16-bit \\ \hline
            Operand Register 2 & T2 & 16-bit \\ \hline
            Input/Output Addressing Register & IOA & 16-bit \\ \hline
        \end{tabular}
    \end{table}
\end{frame}

\begin{frame}
    \frametitle{Memory (M)}
    \begin{itemize}
        \item Address width: 16-bit
        \item Data width: 16-bit
    \end{itemize}
    \begin{circuitikz}
        \draw
        (0,0) node[draw, minimum width=2cm, minimum height=4cm] (chip) {Memory}
        (chip.west) ++(-0.5,1) node[left] {Address (MA)} -- (chip.west |- 0,1)
        (chip.west) ++(-0.5,0.5) node[left] {Memory Input (MI)} -- (chip.west |- 0,0.5)
        (chip.west) ++(-0.5,-0.5) node[left] {Write Enable (WE)} -- (chip.west |- 0,-0.5)
        (chip.west) ++(-0.5,-1) node[left] {Output Enable (OE)} -- (chip.west |- 0,-1)
        (chip.east) ++(0.5,0) node[right] {Memory Output (MO)} -- (chip.east |- 0,0);
    \end{circuitikz}

\end{frame}

\begin{frame}
    \frametitle{Arithmetic Logic Unit (ALU)}
    \begin{circuitikz}
        \draw
        (0,0) node[muxdemux, muxdemux def={Lh=7, NL=2, Rh=4, NR=2, NB=1, NT=2, w=4,
        inset w=1, inset Lh=2, inset Rh=0, square pins=1}] (chip) {ALU}
        % Inputs
        (chip.lpin 1) node[left] {Operand 1 (T1)}
        (chip.lpin 2) node[left] {Operand 2 (T2)}
        (chip.bpin 1) node[below] {Operation (OP)}
        (chip.tpin 1) node[left] {Carry In (CI)}
        (chip.tpin 2) node[right] {Output Enable (OE)}

        % Outputs
        (chip.rpin 1) node[right] {Result (R)}
        (chip.rpin 2) node[right] {Flags (FR)};

    \end{circuitikz}
\end{frame}

\begin{frame}
    \frametitle{Internal Bus}
    \begin{table}[]
        \resizebox{\textwidth}{!}{%
        \begin{tabular}{|l|l|l|l|l|l|l|l|l|l|l|l|}
            \hline
            \textbf{Source} & \textbf{GR} & \textbf{M} & \textbf{T1} & \textbf{T2} & \textbf{IR} & \textbf{PC} & \textbf{IO} & \textbf{IOA} & \textbf{ALU} & \textbf{MA} & \textbf{FR} \\ \hline
            \textbf{GR} & X & X & X & X & - & X & X & - & - & X & - \\ \hline
            \textbf{M} & X & X & X & X & X & X & X & - & - & X & X \\ \hline
            \textbf{T1} & - & - & - & - & - & - & - & - & X & - & - \\ \hline
            \textbf{T2} & - & - & - & - & - & - & - & - & X & - & - \\ \hline
            \textbf{IR} & - & - & - & - & - & - & - & X & - & - & - \\ \hline
            \textbf{PC} & X & X & - & - & - & X & - & - & - & X & - \\ \hline
            \textbf{IO} & X & X & - & - & - & - & - & - & - & - & - \\ \hline
            \textbf{IOA} & - & - & - & - & - & - & X & - & - & - & - \\ \hline
            \textbf{ALU} & X & X & X & X & - & X & - & - & - & X & X \\ \hline
            \textbf{MA} & - & - & - & - & - & - & - & - & - & - & - \\ \hline
            \textbf{FR} & - & X & - & - & - & - & - & - & - & - & - \\ \hline
        \end{tabular}
        }
    \end{table}
    
\end{frame}