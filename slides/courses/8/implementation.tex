\begin{frame}
    \frametitle{Implementation for Lab CPU}
    \textbf{Type of interrupts:}
    \begin{itemize}
        \item Internal interrupts (non-maskable):
        \begin{itemize}
            \item ALU overflow
            \item Software interrupt through the \texttt{INT} instruction
        \end{itemize}
        \item External non-maskable interrupts:
        \begin{itemize}
            \item Lack of Voltage (Power Off)
            \item Special hardware line \texttt{cinm}
        \end{itemize}
        \item External maskable interrupts:
        \begin{itemize}
            \item 8 IO deivce interrupts
            \item Special hardware line \texttt{cintr} for all.
        \end{itemize}
    \end{itemize}
\end{frame}

\begin{frame}
    \frametitle{Implementation of IVT}
    \begin{table}[]
        \resizebox{0.9\textwidth}{!}{%
            \begin{tabular}{|c|c|c|}
                \hline
                \textbf{Address} & \textbf{Name} & \textbf{Type} \\ \hline
                0x0 & Reserved  & \multirow{3}{*}{Internal Interrupt} \\ \cline{1-2}
                0x1 & Software Interrupt (INT) & \\ \cline{1-2}
                0x2 & ALU Overflow & \\ \hline
                0x3 & Lack of Voltage (Power Off) & External Non-Maskable Interrupt \\ \hline
                0x4 & Level 0 & \multirow{8}{*}{External Maskable Interrupt} \\ \cline{1-2}
                0x5 & Level 1 & \\ \cline{1-2}
                0x6 & Level 2 & \\ \cline{1-2}
                0x7 & Level 3 & \\ \cline{1-2}
                0x8 & Level 4 & \\ \cline{1-2}
                0x9 & Level 5 & \\ \cline{1-2}
                0xA & Level 6 & \\ \cline{1-2}
                0xB & Level 7 & \\ \hline
            \end{tabular}
        }
    \end{table}
    \note{
    }

\end{frame}

\begin{frame}
    \frametitle{Interupt Instructions}
    The FR (Flag Register) will have a new bit \texttt{I} which will be used to enable or disable interrupts.
    The following instructions will be added:
    \begin{itemize}
        \item \texttt{EI} - Enable interrupts (sei)
        \item \texttt{DI} - Disable interrupts (cli)
        \item \texttt{INT} - Genrate software interrupt
        \item \texttt{RETI} - Return from interrupt
    \end{itemize}
\end{frame}

\begin{frame}
    \frametitle{Architecture}
    \begin{figure}
        \centering
        \includegraphics[width=0.8\textwidth]{media/isarchitecture.png}
    \end{figure}
\end{frame}

\begin{frame}
    \frametitle{Architecture Signals}
    \begin{itemize}
        \item $ip$ - Software Interrupt
        \item $id$ - ALU Overflow Interrupt
        \item $inm$ - external non-maskable interrupt (comes from $cinm$)
        \item $cintr_{i}$ - external maskable interrupt from device $i$
        \item $ai$ - external maskable interrupt active
        \item $intr$ - external maskable interrupt request to CPU
        \item $sie$/$cie$ - enable/disable external maskable interrupts
    \end{itemize}
\end{frame}

\begin{frame}
    \frametitle{Architecture Components}
    \begin{itemize}
        \item \texttt{REQINT} - Request external maskable interrupt register ($REQINT_{i} = 1$ means that the interrupt $i$ is requested)
        \item \texttt{MEXTINT} - Maskable external interrupt register ($MEXTINT_{i} = 1$ means that the interrupt $i$ is masked)
        \item \texttt{RUNINT} - Current running interrupts register ($RUNINT_{i} = 1$ means that the interrupt on level $i$ is running)
        \item \texttt{Priorrity INT} - Compute if the current interrupt has higher priority than the current running interrupt
        \item \texttt{INTADDR} - Compute the address of the ISR
        \item \texttt{INTPC} - The address for the current ISR
    \end{itemize}
\end{frame}

\begin{frame}
    \frametitle{External maskable interrupts Handling}
    \begin{enumerate}
        \item Verify for external maskable interrupt requests in the \texttt{REQINT} register.
        \item If there is one, verify if external maskable interrupts are enabled. (flag \texttt{I} in FR)
        \item If yes, filter the interrupts that are masked in the \texttt{MEXTINT} register.
        \item Find the highest priority interrupt that is not masked.
        \item Verify ff it has higher priority than the current running interrupts in the \texttt{RUNINT} register:
        \item Send the interrupt to the CPU through the $intr$ signal.
        \item Wait for the CPU to acknowledge the interrupt. ($ai$ signal)
        \item Set the bit in the \texttt{RUNINT} register.
        \item Clear the bit in the \texttt{REQINT} register.
        \item Compute the address of the ISR and set the \texttt{INTPC} register.
    \end{enumerate}
\end{frame}

\begin{frame}
    \frametitle{External non-maskable interrupts Handling}
    \begin{enumerate}
        \item Verify for external non-maskable interrupt requests in the \texttt{inm} signal.
        \item Verify if there is no higher priority interrupt running.
        \item Compute the address of the ISR and set the \texttt{INTPC} register.
    \end{enumerate}
\end{frame}

\begin{frame}
    \frametitle{ALU Overflow Handling}
    \begin{enumerate}
        \item Verify for ALU overflow interrupt requests in the \texttt{id} signal.
        \item Verify if there is no higher priority interrupt running.
        \item Compute the address of the ISR and set the \texttt{INTPC} register.
        \item Inside the ISR, the CPU must clear the  overflow flag in the FR.
    \end{enumerate}
\end{frame}

\begin{frame}
    \frametitle{Software interrupts Handling}
    \begin{enumerate}
        \item Verify for software interrupt requests in the \texttt{ip} signal.
        \item There is no higher priority interrupt running.
        \item Compute the address of the ISR and set the \texttt{INTPC} register.
    \end{enumerate}
\end{frame}

\begin{frame}
    \frametitle{CPU acknowledge interrupts}
    \begin{enumerate}
        \item Save FR on the stack.
        \item Disable external maskable interrupts. (\texttt{EI} can be used to enable them back inside ISR)
        \item Save the current PC on the stack.
        \item Run the jump instruction from the \texttt{INTPC} register address.
    \end{enumerate}
\end{frame}

\begin{frame}
    \frametitle{Exam Questions}
    Template: Having the following values in the registers at time 0,
    interrupts requested with the time of the request and the type of the interrupt,
    and how much time the CPU will take to handle every type of interrupt,
    which is the order of the interrupts that will be handled by the CPU?
\end{frame}


\begin{frame}
    \frametitle{Exam Questions}
    \begin{table}[]
        \begin{tabular}{|c|c|c|c|}
            \hline
            \texttt{REQINT} & \texttt{MEXTINT} & \texttt{RUNINT} & \texttt{FR} \\ \hline
            0x00 & 0x00 & 0x00 & 0x00 \\ \hline
        \end{tabular}
    \end{table}

    \begin{table}[]
        \begin{tabular}{|c|c|c|c|}
            \hline
            Cycles from time 0 & Type of Request & Name \\ \hline
            10 & External Maskable Interupt Level 5 & A \\ \hline
            30 & External Maskable Interupt Level 3 & B \\ \hline
            50 & External Maskable Interupt Level 7 & C \\ \hline
            60 & External Maskable Interupt Level 1 & D \\ \hline
            100 & External Non-Maskable Interupt & E \\ \hline
            140 & ALU Overflow & F \\ \hline
            150 & Software Interupt & G \\ \hline
        \end{tabular}
    \end{table}
\end{frame}



\begin{frame}
    \frametitle{Exam Questions}
    \begin{table}
        \begin{tabular}{|c|c|c|c|}
            \hline
            Type of Request & Cycle to handle \\ \hline
            External Maskable Interupt Level 0 & 30 \\ \hline
            External Maskable Interupt Level 1 & 20 \\ \hline
            External Maskable Interupt Level 2 & 10 \\ \hline
            External Maskable Interupt Level 3 & 40 \\ \hline
            External Maskable Interupt Level 4 & 20 \\ \hline
            External Maskable Interupt Level 5 & 30 \\ \hline
            External Maskable Interupt Level 6 & 10 \\ \hline
            External Maskable Interupt Level 7 & 20 \\ \hline
            External Non-Maskable Interupt & 5 \\ \hline
            ALU Overflow & 10 \\ \hline
            Software Interupt & 10 \\ \hline
        \end{tabular}
    \end{table}
\end{frame}



\begin{frame}
    \frametitle{Exam Questions}
\end{frame}