%\documentclass[notes,usenames,dvipsnames]{beamer}
%\documentclass[notes]{beamer}       % print frame + notes
%\documentclass[notes=only]{beamer}   % only notes
\documentclass[usenames,dvipsnames]{beamer}             % only frames
\usepackage{pgfpages}
%\setbeameroption{show notes on second screen}
%\setbeameroption{show notes}
%subfigures
\usepackage{caption}
\usepackage{subcaption}

%tables packages
\usepackage{multirow}

% math
\usepackage{amsmath}

% bash command
\usepackage{graphicx}
\usepackage{listings}
% varbatim for ascii figures
\usepackage[T1]{fontenc}
\usepackage[utf8]{inputenc}
%\usepackage{verbatim}
\usepackage{lipsum} % for context
\usepackage{fancyvrb}
\usepackage{varwidth}


\newsavebox{\asciigcn}

% notes prefixed in pympress
\addtobeamertemplate{note page}{}{\thispdfpagelabel{notes:\insertframenumber}}
%theme used
\usetheme{Madrid}
%Information to be included in the title page:
\title[Computer Architecture] %optional
{Computer Architecture}

\subtitle{Course no. 3 - Memory}

\author[Ștefan-Dan Ciocîrlan] % (optional, for multiple authors)
{}

\institute[NUSTPB] % (optional)
{
  \inst{}%
  National University of Science and Technology\\
  POLITEHNICA Bucharest
}

\date[NUSTPB 2025] % (optional)
{Computer Architecture}

\logo{
\includegraphics[height=0.9cm]{../../media/LOGO_UNSTPB_en.png}
\includegraphics[height=0.9cm]{../../media/logoACSQ.jpeg}
}


% Roman numerals
\newcommand*{\rom}[1]{\expandafter\@slowromancap\romannumeral #1@}
%split
\usepackage{amsmath}
%colors
%\usepackage[usenames,dvipsnames]{color} %loaded by the dcoument class
%subfigure
\usepackage{subcaption}
% block over block uncover
\setbeamercovered{invisible}
%\setbeamercovered{transparent}

%extra slide content
\AtBeginSection[]
{
  \begin{frame}
    \frametitle{Content}
    \tableofcontents[currentsection]
  \end{frame}
}
%notes or not
%\setbeamertemplate{note page}[plain]

\begin{document}

\frame{\titlepage}

\section{Heirarchy of Memory}
\begin{frame}
    \frametitle{Memory Hierarchy}
    \begin{columns}
        \column{0.5\textwidth}
        \begin{itemize}
            \item CPU make requests to memory for specific addresses.
            \begin{itemize}
                \item Hit: Data found.
                \item Miss: Data not found.
            \end{itemize}
            \item For cache memory:
            \begin{itemize}
                \item cache hit, cache miss.
                \item Hardware manages cache. (L1, L2, L3)
                \item Use block/line as unit of transfer.
            \end{itemize}
            \item For virtual memory:
            \begin{itemize}
                \item page hit, page fault (pault)
                \item OS manages virtual memory.
                \item Use page as unit of transfer.
            \end{itemize}
            \item Memory hierarchy creates the illusion of a large, fast memory.
        \end{itemize}

        \column{0.5\textwidth}
        \newsavebox{\asciimemheir}
        \begin{lrbox}{\asciimemheir}
            \begin{varwidth}{\maxdimen}
            \VerbatimInput[fontsize=\scriptsize]{media/memheir.ascii}
            \end{varwidth}
        \end{lrbox}%

        \begin{figure}[h]
            \centering
            \scalebox{0.7}{\usebox{\asciimemheir}}
        \end{figure}

    \end{columns}
    \note{
        Memory hierarchy creates the illusion of a large, fast memory.
    }
\end{frame}

\begin{frame}
    \frametitle{Memory Locality}
    \begin{itemize}
        \item Temporal locality: If a memory location is referenced, it will tend to be referenced again soon.
        \item Spatial locality: If a memory location is referenced, nearby memory locations will tend to be referenced soon.
    \end{itemize}
    A full cache memory miss depends on the latency (how long to get the first byte) and the bandwidth (how long to get the entire line/block) of the main memory.
\end{frame}

\begin{frame}
    \frametitle{CPU execution time}
    \begin{itemize}
        \item CPU execution time = CPU clock cycles * Clock cycle time (CLKT)
        \item CPU clock cycles = Instruction count (IC) * CPI
        \item CPI = Cycles per instruction
        \item With memory hierarchy: CPU clock cycles = (CPU clock cycles + Memory stall cycles) * Clock cycle time
        \item Memory stall cycles = Number of misses * Miss penalty
        \item Memory stall cycles = IC * Misses per instruction * Miss penalty
        \item Misses per instruction = Miss rate * Memory access per instruction (can be separete for read and write)
    \end{itemize}
\end{frame}

\section{Cache Memory}
\begin{frame}
    \frametitle{Cache Memory Propriety}
    \begin{itemize}
        \item Miss Rate
        \item Average Memory Access Time
        \item Hit Time
        \item Miss Penalty
        \item Cache Size
        \item Block Size
        \item Associativity
    \end{itemize}
\end{frame}

\begin{frame}
    \frametitle{Cache Memory Performence}
    \begin{itemize}
        \item Block Placement
        \item Block Identification
        \item Block Replacement
        \item Write Strategy
    \end{itemize}
\end{frame}

\begin{frame}
    \frametitle{Block Placement}
    \begin{itemize}
        \item Cache memory can cotain a number $n$ of blocks.
        \item RAM memory can contain a number $m$ of blocks, ussualy $m \gg n$.
    \end{itemize}
    There are three main strategies for block placement:
    \begin{itemize}
        \item Direct Mapping (1-way set associative) $pos_{cache}=pos_{ram} \% n$
        \item Fully Associative (n-way set associative) $ \forall \ pos_{cache}$
        \item K-Set Associative (k-way set associative) $\forall pos_{cache} \ \in \text{set } \ pos_{ram} \% \frac{n}{k}$
    \end{itemize}
\end{frame}

\begin{frame}
    \frametitle{Block Placement}
    \note{
        Draw the block placement for each strategy.
    }
\end{frame}

\begin{frame}
    \frametitle{Block Identification}
    \begin{table}[h!]
        \centering
        \begin{tabular}{|p{5cm}|p{3cm}|p{3cm}|}
            \hline
            \multicolumn{3}{|p{11cm}|}{\textbf{Address}} \\
            \hline
            \multicolumn{2}{|p{8cm}|}{\textbf{Block Address}} & \multicolumn{1}{p{3cm}|}{\textbf{Block Offset}} \\
            \hline
            \multicolumn{1}{|p{5cm}|}{\textbf{Tag}} & \multicolumn{1}{|p{3cm}|}{\textbf{Index}} & \multicolumn{1}{p{3cm}|}{\textbf{Block Offset}} \\
            \hline
        \end{tabular}
    \end{table}
    \begin{itemize}
        \item Tag: Identifies the block.
        \item Index: Identifies the set.
        \item Block Offset: Identifies the byte in the block.
        \item There is also a valid bit.
        \item The increase in associativity decreases the number of bits in the index and increases the number of bits in the tag.
        \item The increase in number of sets increase the hardware complexity.
    \end{itemize}
\end{frame}


\begin{frame}
    \frametitle{Block Identification}
    \note{
        Draw the block identification with the TLB and the cache.
    }
\end{frame}

\begin{frame}
    \frametitle{Block Replacement}
    \begin{itemize}
        \item When a block is to be replaced, the cache controller must decide which block to replace.
        \item The replacement policy can be:
        \begin{itemize}
            \item Random
            \item LRU (Least Recently Used) (expensive to implement, partially implemented)
            \item FIFO
        \end{itemize}
    \end{itemize}
\end{frame}

\begin{frame}
    \frametitle{Block Replacement}
    \note{
        Explain the LRU policy.
    }
\end{frame}

\begin{frame}
    \frametitle{Write Strategy}
    \begin{itemize}
        \item Write Through: Write to cache and RAM. (L1 and L2)
        \item Write Back: Write to cache and mark the block as dirty. Write to RAM only when the block is replaced. (L3)
    \end{itemize}
    Allocation policy:
    \begin{itemize}
        \item Write Allocate: Load the block in cache and write to it. (Write Back)
        \item No Write Allocate: Write to RAM only. (Write Through)
    \end{itemize}
    Write buffer (Victim buffer):
    \begin{itemize}
        \item Used to store the write operations.
        \item It is fixed in size. (8-16 entries) When it is full, the CPU must wait.
        \item It is used to avoid the write stall.
        \item Must react when read after write operations are performed on the same block.
    \end{itemize}
\end{frame}

\begin{frame}
    \frametitle{Write Strategy}
    \note{
        Explain the write buffer.
    }
\end{frame}

\begin{frame}
    \frametitle{Exam Question}
    Template: Compute the number of bits for the tag, index and block offset for a cache memory with the following characteristics: X cache size, Y Block Placement, Z block size.

    Example: Compute the number of bits for the tag, index and block offset for a cache memory with the following characteristics: 64KB size, 4-way set associative, 64B block size.
\end{frame}

\begin{frame}
    \frametitle{Type of misses}
    \begin{itemize}
        \item Compulsory (cold start): The first access to a block.
        \item Capacity: The cache is too small.
        \item Conflict: To many blocks map to the same set.
        \item Coherence: The block is modified in another cache.
    \end{itemize}
    \note{
    }
\end{frame}

\begin{frame}
    \frametitle{Type of misses}
    \note{
        Explain the types of misses.
    }
\end{frame}

\begin{frame}
    \frametitle{Cache Memory Benchmark}
    \begin{itemize}
        \item Average Memory Access Time (AMAT) = Hit Time + Miss Rate * Miss Penalty
        \item Hit Time = Time to access the cache.
        \item Miss Penalty = Time to access the main memory.
        \item Miss Rate = Number of misses / Number of accesses.
        \item CPU execution Time
        \item Power consumption
    \end{itemize}
\end{frame}

\section{Virtual Memory}
\begin{frame}
    \frametitle{Virtual Memory}
    Benefits of Virtual Memory:
    \begin{itemize}
        \item Efficient memory management
        \item Enhanced protection
        \item Shared memory capabilities
        \item Process relocation
        \item Faster process creation and startup time (without loading the entire process into memory)
    \end{itemize}
    
    The process of obtaining the physical address from the virtual address is known as translation.
    The operating system manages the virtual memory.
\end{frame}

\begin{frame}
    \frametitle{Allocation Policies}
    \begin{itemize}
        \item\textbf{Paged Virtual Memory}: In this approach, the virtual memory is divided into fixed-size pages.
        \item \textbf{Segmented Virtual Memory}: Here, the virtual memory is divided into segments of varying sizes
        based on logical divisions.
        \item \textbf{Combined Virtual Memory}: This method divides the virtual memory into segments, with each segment's
        size being a multiple of the page size.
    \end{itemize}
    The miss penalty refers to the time it takes to retrieve a page from the disk. Therefore, the operating system
    aims to minimize the number of misses by adopting a fully associative design, allowing pages to be placed
    anywhere in the main memory.
\end{frame}

\begin{frame}
    \frametitle{TLB vs MMU}

    \begin{itemize}
        \item \textbf{Translation Lookaside Buffer (TLB)}
        \begin{itemize}
            \item A specialized cache designed to speed up the translation from virtual to physical addresses.
            \item Stores recent translations for quick access.
            \item Checked first during the address translation process.
            \item Very fast, but limited in size.
        \end{itemize}
        \item \textbf{Memory Management Unit (MMU)}
        \begin{itemize}
            \item A hardware component responsible for managing memory and caching operations.
            \item Translates virtual addresses to physical addresses using page tables.
            \item Manages page faults and enforces memory protection.
            \item More complex and integrated directly into the CPU.
        \end{itemize}
    \end{itemize}
\end{frame}


\begin{frame}
    \frametitle{TLB vs MMU}
    \begin{table}[h!]
        \centering
        \begin{tabular}{|p{2cm}|p{4cm}|p{4cm}|}
            \hline
            \textbf{Component} & \textbf{Function} & \textbf{Characteristics} \\
            \hline
            TLB & Caches recent address translations & Fast, limited size \\
            \hline
            MMU & Manages memory and performs address translations & Complex, integrated into the CPU \\
            \hline
        \end{tabular}
    \end{table}
\end{frame}

\begin{frame}
    \frametitle{Exam Questions}
    \begin{itemize}
        \item What happens to CPU execution time if we increase the block size to reduce the miss rate?
        \item What happens to CPU execution time if we increase the cache size to reduce the miss rate?
        \item What happens to CPU execution time if we increase the associativity to reduce the miss rate?
        \item What happens to CPU execution time if we add multiple levels of cache?
    \end{itemize}
\end{frame}

\begin{frame}
    \frametitle{Block size increase}
    We have a cache of size $n$ and a block size of $b_{0}$.
    The miss rate is $m_{0}$ and the miss penalty is $p_{access}$ plus $p_{byte}$
    for each byte in clock cycles. If we increase the block size to $b_{1}$,
    the miss rate will be $m_{1}$. Compute Average memory access time (AMAT)
    for both cases. The hit time is $h$.
\end{frame}
\begin{frame}
    \frametitle{Block size increase}
\end{frame}


\begin{frame}
    \frametitle{Cache size increase}
    We have a block size of $b_{0}$ and a miss rate of $m_{0}$.
    The miss penalty is $p_{access}$ plus $p_{byte}$ for each byte in clock cycles.
    If we increase the cache size to $n_{1}$, the miss rate will be $m_{1}$.
    Compute Average memory access time (AMAT) for both cases.
    The hit time is $h_{0}$ for the first cache and $h_{1}$ for the second one.
\end{frame}
\begin{frame}
    \frametitle{Cache size increase}
\end{frame}

\begin{frame}
    \frametitle{Associativity increase}
    We have a cache of size $n$ and a block size of $b_{0}$ with a $k_{0}$-way associativity.
    The miss rate is $m_{0}$, and the miss penalty is $p_{access}$ plus $p_{byte}$ for
    each byte in clock cycles. If we increase the associativity to $k_{1}$,
    the miss rate will be $m_{1}$, and will increase the clock cycle time from $c_{0}$ to $c_{1}$.
    Hit time is $h$ for both. Compute Average memory access time (AMAT) for both cases.
\end{frame}
\begin{frame}
    \frametitle{Associativity increase}
\end{frame}

\begin{frame}
    \frametitle{Multiple levels of cache}
    Suppose that in $m$ memory references, there are $m_{1}$ misses in the first cache and $m_{2}$ misses in the second cache.
    The hit time for the first cache is $h_{1}$, and for the second cache is $h_{2}$.
    The miss penalty for the second cache is $p_{2}$.
    Compute the Average memory access time (AMAT).
\end{frame}
\begin{frame}
    \frametitle{Multiple levels of cache}
\end{frame}

%\section{Image Representation Systems}
%\input{image.tex}

%\section{Audio Representation Systems}
%\input{audio.tex}

\section{Q\&A}
\begin{frame}
\end{frame}

%\begin{frame}
%\frametitle{Table of Contents}
%\tableofcontents
%\end{frame}


\end{document}