%\documentclass[notes,usenames,dvipsnames]{beamer}
%\documentclass[notes]{beamer}       % print frame + notes
%\documentclass[notes=only]{beamer}   % only notes
\documentclass[usenames,dvipsnames]{beamer}             % only frames
\usepackage{pgfpages}
%\setbeameroption{show notes on second screen}
%\setbeameroption{show notes}
%subfigures
\usepackage{caption}
\usepackage{subcaption}

%tables packages
\usepackage{multirow}

% math
\usepackage{amsmath}

% bash command
\usepackage{graphicx}
\usepackage{listings}
% varbatim for ascii figures
\usepackage[T1]{fontenc}
\usepackage[utf8]{inputenc}
%\usepackage{verbatim}
\usepackage{lipsum} % for context
\usepackage{fancyvrb}
\usepackage{varwidth}


\newsavebox{\asciigcn}

% notes prefixed in pympress
\addtobeamertemplate{note page}{}{\thispdfpagelabel{notes:\insertframenumber}}
%theme used
\usetheme{Madrid}
%Information to be included in the title page:
\title[Computer Architecture] %optional
{Computer Architecture}

\subtitle{Course no. 3 - Memory}

\author[Ștefan-Dan Ciocîrlan] % (optional, for multiple authors)
{}

\institute[NUSTPB] % (optional)
{
  \inst{}%
  National University of Science and Technology\\
  POLITEHNICA Bucharest
}

\date[NUSTPB 2025] % (optional)
{Computer Architecture}

\logo{
\includegraphics[height=0.9cm]{../../media/LOGO_UNSTPB_en.png}
\includegraphics[height=0.9cm]{../../media/logoACSQ.jpeg}
}


% Roman numerals
\newcommand*{\rom}[1]{\expandafter\@slowromancap\romannumeral #1@}
%split
\usepackage{amsmath}
%colors
%\usepackage[usenames,dvipsnames]{color} %loaded by the dcoument class
%subfigure
\usepackage{subcaption}
% block over block uncover
\setbeamercovered{invisible}
%\setbeamercovered{transparent}

%extra slide content
\AtBeginSection[]
{
  \begin{frame}
    \frametitle{Content}
    \tableofcontents[currentsection]
  \end{frame}
}
%notes or not
%\setbeamertemplate{note page}[plain]

\begin{document}

\frame{\titlepage}

\section{Heirarchy of Memory}
\begin{frame}
    \frametitle{Memory Hierarchy}
    \begin{columns}
        \column{0.5\textwidth}
        \begin{itemize}
            \item The CPU makes requests to memory for specific addresses.
            \begin{itemize}
                \item Hit: Data is found.
                \item Miss: Data is not found.
            \end{itemize}
            \item For cache memory:
            \begin{itemize}
                \item Cache hit, cache miss.
                \item Hardware manages cache levels (L1, L2, L3).
                \item Uses blocks or lines as units of transfer.
            \end{itemize}
            \item For virtual memory:
            \begin{itemize}
                \item Page hit, page fault.
                \item The operating system manages virtual memory.
                \item Uses pages as units of transfer.
            \end{itemize}
        \end{itemize}

        \column{0.5\textwidth}
        \newsavebox{\asciimemheir}
        \begin{lrbox}{\asciimemheir}
            \begin{varwidth}{\maxdimen}
            \VerbatimInput[fontsize=\scriptsize]{media/memheir.ascii}
            \end{varwidth}
        \end{lrbox}%

        \begin{figure}[h]
            \centering
            \scalebox{0.7}{\usebox{\asciimemheir}}
        \end{figure}

    \end{columns}
    \note{
        The memory hierarchy creates the illusion of a large, fast memory system.
    }
\end{frame}

\begin{frame}
    \frametitle{Memory Locality}
    The memory hierarchy creates the illusion of a large, fast memory system.
    \begin{itemize}
        \item \textbf{Temporal locality:} If a memory location is accessed, it is likely to be accessed again soon.
        \item \textbf{Spatial locality:} If a memory location is accessed, nearby memory locations will likely be accessed soon after.
    \end{itemize}
    A full cache memory miss depends on two factors: the latency of the main memory (how long it takes to retrieve the first byte) and the bandwidth (how long it takes to retrieve the entire line or block).
\end{frame}

\begin{frame}
    \frametitle{CPU Execution Time}
    \begin{itemize}
        \item \textbf{CPU execution time} = CPU clock cycles $\times$ Clock cycle time (CLKT)
        \item \textbf{CPU clock cycles} = Instruction count (IC) $\times$ Cycles per instruction (CPI)
        \item \textbf{With memory hierarchy:} CPU clock cycles = (CPU clock cycles $+$ Memory stall cycles) $\times$ Clock cycle time
        \item \textbf{Memory stall cycles} = Number of misses $\times$ Miss penalty
        \item \textbf{Memory stall cycles} = IC $\times$ Misses per instruction $\times$ Miss penalty
        \item \textbf{Misses per instruction} = Miss rate $\times$ Memory accesses per instruction (this can be separated for read and write operations)
    \end{itemize}
\end{frame}

\section{Cache Memory}
\begin{frame}
    \frametitle{Cache Memory Propriety}
    \begin{itemize}
        \item Miss Rate
        \item Average Memory Access Time
        \item Hit Time
        \item Miss Penalty
        \item Cache Size
        \item Block Size
        \item Associativity
    \end{itemize}
\end{frame}

\begin{frame}
    \frametitle{Cache Memory Performence}
    \begin{itemize}
        \item Block Placement
        \item Block Identification
        \item Block Replacement
        \item Write Strategy
    \end{itemize}
\end{frame}

\begin{frame}
    \frametitle{Block Placement}
    \begin{itemize}
        \item Cache memory contains a number $n$ of blocks.
        \item RAM memory contains a number $m$ of blocks, ussualy $m \gg n$.
    \end{itemize}
    There are three main strategies for block placement:
    \begin{itemize}
        \item Direct Mapping (1-way set associative):\\
        $p_{\text{cache}}=p_{\text{ram}} \mod n$
        \item Fully Associative (n-way set associative):\\
        $ \forall \ p_{\text{cache}}$
        \item K-Set Associative (k-way set associative):\\
        $\forall p_{\text{cache}} \ \in \text{set } (p_{\text{ram}} \mod \frac{n}{k})$
    \end{itemize}
\end{frame}

% \begin{frame}
%     \frametitle{Block Placement}
%     \note{
%     Draw the block placement for each strategy.
%     }
% \end{frame}

\begin{frame}
    \frametitle{Block Identification}
    \begin{table}[h!]
        \centering
        \begin{tabular}{|p{5cm}|p{3cm}|p{3cm}|}
            \hline
            \multicolumn{3}{|p{11cm}|}{\textbf{Address}} \\
            \hline
            \multicolumn{2}{|p{8cm}|}{\textbf{Block Address}} & \multicolumn{1}{p{3cm}|}{\textbf{Block Offset}} \\
            \hline
            \multicolumn{1}{|p{5cm}|}{\textbf{Tag}} & \multicolumn{1}{|p{3cm}|}{\textbf{Index}} & \multicolumn{1}{p{3cm}|}{\textbf{Block Offset}} \\
            \hline
        \end{tabular}
    \end{table}
    \begin{itemize}
        \item \textbf{Tag}: Identifies the block in memory.
        \item \textbf{Index}: Identifies the set within the cache.
        \item \textbf{Block Offset}: Specifies the exact byte within the block.
        \item A valid bit is also included to indicate if the block is valid.
        \item Increasing the associativity reduces the number of bits needed for the index but increases the bits required for the tag.
        \item Increasing the number of sets raises the complexity of the hardware.
    \end{itemize}
\end{frame}

% \begin{frame}
%     \frametitle{Block Identification}
%     \note{
%         Illustrate the block identification process involving the TLB and the cache.
%     }
% \end{frame}

\begin{frame}
    \frametitle{Block Replacement}
    \begin{itemize}
        \item When a block needs to be replaced, the cache controller must determine which block to evict.
        \item The replacement policy can be one of the following:
        \begin{itemize}
            \item Random
            \item LRU (Least Recently Used) - this can be expensive to implement and is often only partially implemented
            \item FIFO (First In, First Out)
        \end{itemize}
    \end{itemize}
\end{frame}

% \begin{frame}
%     \frametitle{Block Replacement}
%     \note{
%         Explain the LRU (Least Recently Used) policy.
%     }
% \end{frame}

\begin{frame}
    \frametitle{Write Strategy}
    \textbf{Write Strategy:}
    \begin{itemize}
        \item \textbf{Write Through:} Write to both the cache and RAM (L1 and L2).
        \item \textbf{Write Back:} Write to the cache and mark the block as dirty. Write to RAM only when the block is replaced (L3).
    \end{itemize}
    
    \textbf{Allocation Policy:}
    \begin{itemize}
        \item \textbf{Write Allocate:} Load the block into the cache and write to it (used with Write Back).
        \item \textbf{No Write Allocate:} Write directly to RAM only (used with Write Through).
    \end{itemize}
\end{frame}

\begin{frame}
    \frametitle{Write Buffer}
    \textbf{Write Buffer (Victim Buffer):}
    \begin{itemize}
        \item Stores write operations.
        \item Has a fixed size (usually 8-16 entries). When full, the CPU must wait.
        \item Helps avoid write stalls.
        \item Must respond correctly when read-after-write operations access the same block.
    \end{itemize}
\end{frame}

% \begin{frame}
%     \frametitle{Write Strategy}
%     \note{
%         Explain the write buffer.
%     }
% \end{frame}

% \begin{frame}
%     \frametitle{Exam Question}
%     Template: Calculate the number of bits for the tag, index, and block offset for a cache memory with the following characteristics: X cache size, Y block placement, and Z block size.

%     Example: Calculate the number of bits for the tag, index, and block offset for a cache memory with the following characteristics: 64KB size, 4-way set associative, 64B block size.
% \end{frame}

\begin{frame}
    \frametitle{Types of Misses}
    \begin{itemize}
        \item Compulsory (cold start): The first access to a block.
        \item Capacity: Occurs when the cache is too small.
        \item Conflict: Happens when too many blocks map to the same set.
        \item Coherence: Arises when another cache modifies the block.
    \end{itemize}
    \note{
    }
\end{frame}

% \begin{frame}
%     \frametitle{Types of Misses}
%     \note{
%         Explain the different types of cache misses.
%     }
% \end{frame}

\begin{frame}
    \frametitle{Cache Memory Benchmark}
    \begin{itemize}
        \item Average Memory Access Time (AMAT) = Hit Time + Miss Rate $\times$ Miss Penalty
        \item Hit Time: The time taken to access the cache.
        \item Miss Penalty: The time taken to access the main memory.
        \item Miss Rate: The ratio of misses to the total number of accesses.
        \item CPU Execution Time
        \item Power Consumption
    \end{itemize}
\end{frame}

\section{Virtual Memory}
\begin{frame}
    \frametitle{Virtual Memory}
    Benefits of Virtual Memory:
    \begin{itemize}
        \item Efficient memory management
        \item Enhanced protection
        \item Shared memory capabilities
        \item Process relocation
        \item Faster process creation and startup time (without loading the entire process into memory)
    \end{itemize}
    
    The process of obtaining the physical address from the virtual address is known as translation.
    The operating system manages the virtual memory.
\end{frame}

\begin{frame}
    \frametitle{Allocation Policies}
    \begin{itemize}
        \item\textbf{Paged Virtual Memory}: In this approach, the virtual memory is divided into fixed-size pages.
        \item \textbf{Segmented Virtual Memory}: Here, the virtual memory is divided into segments of varying sizes
        based on logical divisions.
        \item \textbf{Combined Virtual Memory}: This method divides the virtual memory into segments, with each segment's
        size being a multiple of the page size.
    \end{itemize}
    The miss penalty refers to the time it takes to retrieve a page from the disk. Therefore, the operating system
    aims to minimize the number of misses by adopting a fully associative design, allowing pages to be placed
    anywhere in the main memory.
\end{frame}

\begin{frame}
    \frametitle{TLB vs MMU}

    \begin{itemize}
        \item \textbf{Translation Lookaside Buffer (TLB)}
        \begin{itemize}
            \item A specialized cache designed to speed up the translation from virtual to physical addresses.
            \item Stores recent translations for quick access.
            \item Checked first during the address translation process.
            \item Very fast, but limited in size.
        \end{itemize}
        \item \textbf{Memory Management Unit (MMU)}
        \begin{itemize}
            \item A hardware component responsible for managing memory and caching operations.
            \item Translates virtual addresses to physical addresses using page tables.
            \item Manages page faults and enforces memory protection.
            \item More complex and integrated directly into the CPU.
        \end{itemize}
    \end{itemize}
\end{frame}


\begin{frame}
    \frametitle{TLB vs MMU}
    \begin{table}[h!]
        \centering
        \begin{tabular}{|p{2cm}|p{4cm}|p{4cm}|}
            \hline
            \textbf{Component} & \textbf{Function} & \textbf{Characteristics} \\
            \hline
            TLB & Caches recent address translations & Fast, limited size \\
            \hline
            MMU & Manages memory and performs address translations & Complex, integrated into the CPU \\
            \hline
        \end{tabular}
    \end{table}
\end{frame}

\begin{frame}
    \frametitle{Exam Questions}
    \begin{itemize}
        \item What happens to CPU execution time if we increase the block size to reduce the miss rate?
        \item What happens to CPU execution time if we increase the cache size to reduce the miss rate?
        \item What happens to CPU execution time if we increase the associativity to reduce the miss rate?
        \item What happens to CPU execution time if we add multiple levels of cache?
    \end{itemize}
\end{frame}

\begin{frame}
    \frametitle{Block size increase}
    We have a cache of size $n$ and a block size of $b_{0}$.
    The miss rate is $m_{0}$ and the miss penalty is $p_{access}$ plus $p_{byte}$
    for each byte in clock cycles. If we increase the block size to $b_{1}$,
    the miss rate will be $m_{1}$. Compute Average memory access time (AMAT)
    for both cases. The hit time is $h$.
\end{frame}
\begin{frame}
    \frametitle{Block size increase}
\end{frame}


\begin{frame}
    \frametitle{Cache size increase}
    We have a block size of $b_{0}$ and a miss rate of $m_{0}$.
    The miss penalty is $p_{access}$ plus $p_{byte}$ for each byte in clock cycles.
    If we increase the cache size to $n_{1}$, the miss rate will be $m_{1}$.
    Compute Average memory access time (AMAT) for both cases.
    The hit time is $h_{0}$ for the first cache and $h_{1}$ for the second one.
\end{frame}
\begin{frame}
    \frametitle{Cache size increase}
\end{frame}

\begin{frame}
    \frametitle{Associativity increase}
    We have a cache of size $n$ and a block size of $b_{0}$ with a $k_{0}$-way associativity.
    The miss rate is $m_{0}$, and the miss penalty is $p_{access}$ plus $p_{byte}$ for
    each byte in clock cycles. If we increase the associativity to $k_{1}$,
    the miss rate will be $m_{1}$, and will increase the clock cycle time from $c_{0}$ to $c_{1}$.
    Hit time is $h$ for both. Compute Average memory access time (AMAT) for both cases.
\end{frame}
\begin{frame}
    \frametitle{Associativity increase}
\end{frame}

\begin{frame}
    \frametitle{Multiple levels of cache}
    Suppose that in $m$ memory references, there are $m_{1}$ misses in the first cache and $m_{2}$ misses in the second cache.
    The hit time for the first cache is $h_{1}$, and for the second cache is $h_{2}$.
    The miss penalty for the second cache is $p_{2}$.
    Compute the Average memory access time (AMAT).
\end{frame}
\begin{frame}
    \frametitle{Multiple levels of cache}
\end{frame}

%\section{Image Representation Systems}
%\input{image.tex}

%\section{Audio Representation Systems}
%\input{audio.tex}

\section{Q\&A}
\begin{frame}
\end{frame}

%\begin{frame}
%\frametitle{Table of Contents}
%\tableofcontents
%\end{frame}


\end{document}